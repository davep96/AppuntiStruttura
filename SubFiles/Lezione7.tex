\documentclass[../AppuntiStruttura]{subfiles}


\begin{document}
	\section*{Lezione 7, 19 Marzo 2018}
	Gli argomenti di oggi copriranno la correzione relativistica all'energia cinetica e lo spettro degli atomi idrogenoidi in presenza di campo magnetico. Siamo quindi nell'ambito della struttura fine e successivamente iperfine dell'atomo di idrogeno. Abbiamo visto l'introduzione del termine di correzione spin orbita, e abbiamo visto come questa è una correzione relativistica (che si può vedere anche dal termine moltiplicativo inversamente proporzionale al quadrato della velocità della luce). Vediamo ora la correzione relativistica ad un altro termine, quello cinetico. Quello che facciamo è esprimere l'energia cinetica nell'ambito della relatività ristretta. Possiamo considerare l'energia totale, che si può scrivere come energia a riposo più in contributo cinetico. Si può dimostrare che questa può essere scritta come:
	\begin{gather}
		K=\sqrt{\mu_{0}^{2}c^{4}+p^{2}c^{2}}-\mu c^{2}
	\end{gather}
	La massa nella formula è la massa relativistica si ha che, con $ p=\mu v=\gamma\mu_{0}v $:
	\begin{gather}
		\sqrt{\mu_{0}^{2}c^{4}+p^{2}c^{2}}=\sqrt{\mu_{0}^{2}c^{4}+\left(\gamma\mu_{0}v\right)^{2}c^{2}}
	\end{gather}
	Facendo lo sviluppo di taylor dell'energia in $ \frac{1}{c^{2}} $ si ha, all'ordine zero l'espressione classica dell'energia, mentre al primo ordine si ha la correzione relativistica:
	\begin{gather}
		K=\mu_{0}c^{2}\left(1+\dfrac{1}{2}\dfrac{p^{2}}{\mu^{2}c^{2}}-\dfrac{1}{8}\dfrac{p^{2}}{\mu_{0}^{4}c^{4}}+\dots-1\right)
	\end{gather}
	Bisogna calcolare quindi, in tutt l'elemento di matrice:
	\begin{gather}
	\bra{n,l}\dfrac{p^{2}}{8\mu^{3}c^{2}}\ket{n,l}=\dots=-E_{n}^{0}\dfrac{Z^{2}\alpha^{2}}{n^{2}}\left(\dfrac{3}{4}-\dfrac{n}{l+\frac{1}{2}}\right)\left(\dfrac{\mu_{0}}{m_{e}}\right)^{3}
	\end{gather}
	Quest'ultima espressione è dipendente dall'energia imperturbata dell'atomo di idrogeno. Gli stati imperturbati non sono tuttavia autostati dell'operatore $ p^{4} $ ma noi ci limitiamo al calcolo di elementi diagonali. Il bra e il ket sono lo stesso stato perché sono i termini più rilevanti nello sviluppo perturbativo. Infatti i termini correttivi sono tanto più piccoli quanto più grandi quanto maggiore è la differenza tra i valori di energia. Il valore di aspettazione di questi è trascurabile rispetto al contributo degli operatori sulla diagonale. Vediamo che è una correzione all'energia proporzionale a $ \alpha^{2} $ con un  $ c $ al denominatore quindi va al secondo ordine nella correzione relativistica. Poiché è dello stesso ordine della correzione spin orbita non va trascurato. Tipicamente, l'interazione spin orbita genera però uno splitting quindi ha un effetto maggiore. Questo termine di correzione rappresenta solo uno shifting dei livelli, senza disturbare troppo il sistema. Nel caso di atomi idrogenoidi però la situazione diventa più interessante?. Consideriamo adesso la hamiltoniana relativistica (non propriamente vanno altri termini che aggiungeremo dopo). Questa è data da:
	\begin{gather}
		H_{rel}=H_{S-O}-\dfrac{p^{4}}{8\mu^{3}c^{2}}
	\end{gather}
	Si ha che:
	\begin{gather}
	\label{eqn:corrreleig}
	\bra{nlj}H_{rel}\ket{nlj}=-\dfrac{Z^{4}\alpha^{2}}{n^{3}}E_{Ha}\left(\dfrac{\mu_{0}}{m_{e}}\right)^{3}\left(\dfrac{1}{2j+1}-\dfrac{3}{8n}\right)
	\end{gather}
	Abbiamo incluso una correzione relativistica all'energia cinetica che vale a tutti i livelli ma l'abbiamo costruita solo per gli stati con $ l\neq0 $. Si può dimostrare che questa correzione vale anche per il caso $ l=0 $. Questo però è valido solamente aggiungendo il termine correttivo:
	\begin{gather}
	H_{D}=\text{Costante}\delta(\vec{r})
	\end{gather}
	La costante si trova sul libro. Il valore di aspettazione è dunque proporzionale al modulo quadro della funzione d'onda nello zero. Questo, per $ l\neq0 $ è nullo. Si ha quindi che per $ l\neq0 $ questo termine è ininfluente. In conclusione l'espresione \ref{eqn:correleig} è valida per ogni $ l $ ma si trovai in modo diverso a seconda del caso di $ l=0 $ o $ l\neq 0 $. Si può dimostrare che questa è anche la soluzione dell'equazione di Dirac e quindi è corretta ad ogni ordine. Quello che ci aspettiamo è che due livelli con lo stesso $ j $ a parità di $ n $ hanno la stessa energia. 
	
	Tenendo conto delle regole di selezione è quindi possibile fare un grafico delle transizioni possibili con le corrispondenti linee. Per ottenere lo spettro sperimentalmente è necessario fare degli accorgimenti,  perché le linee sono sempre allargati. I motivi dell'allargamento sono già stati riportati. Per evitare gli allargamenti dovuti all'effetto doppler, che risulta essere maggiore della risoluzione necessaria per vedere alcuni degli splitting spin orbita.
	
	Risulta essere presente anche un altro termine di correzione, dovuto a fenomeni relativistici di fluttuazioni del campo elettromagnetico. Con questa correzione i risultati sperimentali coincidono meglio con la teoria. 
	
	Vediamo adesso cosa succede nel momento in cui rompiamo la simmetria sferica del problema e cerchiamo di estrarre più informazioni eliminando alcune delle degenerazion iancora presenti nel sistema. Si è visto che un certo approccio i livelli 3nergetici non dipendono da $ l $, poi con spin orbita dipendono anche da $ l $, successivamente non più da $ l $ ma da $ j $\dots. Rompiamo ora la simmetria sferica del sistema per far dipendere l'energia anche da $ l_{z} $. Questo in genere si fa inserendo un campo magnetico. Il momento magnetico atomico è dato dalla somma dei due contributi, di spin e orbitale. Di fatto sappiamo che:
	\begin{gather}
	\vec{\mu}=-\dfrac{\mu_{B}}{\hbar}\left(\vec{L}+2\vec{S}\right)
	\end{gather}
	Si ha quindi un contributo all'Hamiltoniana di:
	\begin{gather}
		H_{magn}=-\scal{\vec{B}}{\vec{\mu}}
	\end{gather}
	che rappresenta il contributo magnetico. Questo termine non è diagonale in $ \ket{l,s,j,m_{j}} $ ma lo è in $ \ket{l,s,m_{s},m_{s}} $. A seconda della scelta della base scelta possiamo calcolare o $ H_{magn} $ o $ H_{S-O} $.  Consideriamo ora i due casi, campo debole e campo forte. Nel caso di campo forte supponiamo di poter trascorare il termine di spin orbita. In questo caso, l'energia magnetica è data adll'espressione:
	$$ E_{magn}=\bra{m_{l},m_{s}}H_{mang}\ket{m_{l},m_{s}} $$
	Questi momenti angolari si allineerano al campo magnetico per dare un contributo il più piccolo possibile all'energia. Viceversa, quando il campo è debole, ossia quello che avviene normalmente, la situazione è diversa. I livelli sono determinati dall'interazione spin orbita e questi livelli saranno a loro volta splittati a causa della presenza del campo magnetico secondo le loro componenti dirette secondo la proiezione del momento angolare sulla direzione del campo. In questo casoi l $ j $ totale tende ad allinearsi. $ S $ e $ L $ hanno una loro posizione relativa fissata dalla loro interazione spin orbita e precedono attorno a $ J $. Quello che si può pensare è che in media nel tempo la direzione di $ L $ è parallela a $ J $ e stessa cosa vale anche per $ S $. Sostanzialmetne i momenti magnetici $ L,\ J,\ S $ sono in media diretti nella stessa direzione, quella del campo magnetico esterno $ B $. Il valore finale è:
	\begin{gather}
		E_{mang}\approx\expval{B}{j,m_{j}}=g_{l}\mu_{B}Bm_{j}
	\end{gather}
	dove $ g $ è il \emph{fattore di Landé}. Questo fattore è stato calcolato in classe ma non è riportato, la dimostrazione è sul libro. Si ha che:
	\begin{gather}
	g_{l}=1+\dfrac{j\left(j+1\right)+s\left(s+1\right)-l\left(l+1\right)}{2j\left(j+1\right)}
	\end{gather}
	Quindi questo termine magnetico è causa di un ulteriore splitting dovuto al numero quantico $ m_{j} $. Questo effetto è noto come effetto Zeeman. Nel caso di campi che tendono a zero gli spin orbita sono ben chiari, ma non c'è splitting di $ m_{j} $. Aumentando man mano il campo, questa singola lineaa viene splittata, rompendo la degenerazione. Per campi grandi questa separazione è netta. Inoltre pè anche possibile avere un overlapping che confonde i le linee spettrali appartenenti a valori di $ J $ totale differenti. 
	
	Passiamo ora ad introdurre i sistemi e nuclei a più particelle. Fino ad offi abbiamo visto un sistema che contiene un sistema ad un solo elettrone. Consideriamo ora il caso in cui ci siano pià elettroni. Il punto chiave per affrontare ed intepretare le evidenze perimentali e cosruire la teoria è il concetto di indistinguivilità delle particelle. Questa indistinguibilità, non è ovvia. Questo argomento è stato largamente affrontato nel corso di meccanica quantistica e quindi non verrà riportato in questo testo.
 \end{document}